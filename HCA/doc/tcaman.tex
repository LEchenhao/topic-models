\documentclass[a4paper,english]{article}
\usepackage{a4wide}
\usepackage[latin1]{inputenc}
\usepackage{babel}
\usepackage{verbatim}

\usepackage[fancyhdr]{latex2man}
\usepackage{natbib}
\bibliographystyle{agsm}

\setDate{2014/5/22}
\setVersion{0.4}

\begin{document}

\begin{Name}{1}{tca}{Wray Buntine \\ Jinjing Li}{Data Analysis Tools}{tca}

  \Prog{tca} is a C implementation of the non-parametric dynamic topic model from \citet{dtmpypwl}. It allows both the topic mixtures and the word-topic distributions to change over epochs (time) through two-parameter Poisson-Dirichlet processes (PDP). The software can also be configured to run as a standard LDA model with proper parameters, though this is intended only for testing. ``Document completion'' testing, coherence measurements with PMI  and a few other diagnostic values are supported.  \Prog{tca} can run multi-core to speed up the computation.

\end{Name}

\section{Synopsis}
%%%%%%%%%%%%%%%%%%

\Prog{tca} \oOpt{-?} \oOptArg{-?}{Arg}
                 \Arg{DataStem} \Arg{RepStem}

\section{Description}
%%%%%%%%%%%%%%%%%%%%%
\Prog{tca} reads the collection of files with stem
\Arg{DataStem} that form the input set of data.
When checkpointing, or at termination, the output is written
to files with stem  \Arg{RepStem}.
On restart with the \Opt{-r} option, some of these
are also read initially to restore the previous state.
A log of the run is reported to \File{stderr} if the
\Opt{-e} option is used.  By default, the log goes to
\File{RepStem.log}.

The programme is research software, so not all options
or combinations of options work correctly.
Note that in this release, all the more experimental features
have been stripped, so this release contains
only moderately well tested components.

The programme runs a Gibbs sampler for a model consisting of the following three parts:
\begin{Description}[model]
\item[theta:] this is the prior on topic vector (theta) for each document.
\Prog{tca} uses a Poisson-Dirichlet distribution with parameters $a_{\theta}$ and  $b_{\theta}$ .
The vector has dimension $T$ (the number of topics).

\item[phi:] this is the prior on word vector  (phi) for each topic. 
\Prog{tca} uses a Poisson-Dirichlet distribution with parameters $a_{\phi}$ and  $b_{\phi}$ .
The vector has dimension $W$ (the number of words).

\item[mu:] this is the prior on topic vector (mu) for each epoch.
\Prog{tca} uses a Poisson-Dirichlet distribution with parameters $a_{\mu}$ and  $b_{\mu}$ .
The vector has dimension $T$ (the number of topics).

\item[burst:]  this is the burstiness component which has
a document specific variant of the word vector for
each topic.  This is not used by default.
\end{Description}

It is possible to find-tune the model specification through the
\Opt{-S} option.

The programme uses generalised second order Stirling numbers
with the library extracted from \Prog{libstb} version 1.8
released at \URL{https://github.com/wbuntine/libstb}.
This is initialised with predefined bounds on the tables,
and these can be modified with the \Opt{-N} option.
This should be used for collections with more than
20,000 documents, but its best to run first and on
error, increase the bounds.


\section{Options}
%%%%%%%%%%%%%%%%%

Options have a single letter followed by a possible
single argument. Options are grouped under
the following functions:
\emph{setting of hyperparameters}, 
\emph{controlling sampling of hyperparameters},
\emph{general control}, and
\emph{testing and reports}

\subsection{Setting of hyperparameters}
\begin{Description}[\OptArg{-t}{transfile}]\setlength{\itemsep}{0cm}
\item[\OptArg{-S}{var=value}]  Set variable \texttt{var} to float \texttt{value},
where \texttt{var} can be one of:
\begin{Description}[bdk]
\item[am] discount parameter for 
\texttt{mu}, the topic distribution.
\item[at] discount parameter for 
\texttt{theta}, topic distribution per document.
\item[ap0] discount parameter for 
\texttt{phi}, the word distribution per topic, in the first epoch.
\item[ap1] discount parameter for 
\texttt{phi} for the second epoch onwards.

\item[bm0] concentration parameter for 
\texttt{mu} in the first epoch.
\item[bm1] concentration parameter for
\texttt{mu} for the second epoch onwards.
\item[bt] concentration parameter for 
\texttt{theta}
\item[b0p] concentration parameter for
  on the prior of \texttt{phi}.
\item[b0m] concentration parameter for
on the prior of \texttt{mu}
\item[bp0] concentration parameter for
  \texttt{phi} in the first epoch.
\item[bp1] concentration parameter for
  \texttt{phi} for the second epoch onwards.

\item[ab] discount parameter for burstiness
\item[bb] concentration parameter for burstiness, a constant initially
     but subsequent sampling will allow a different value per topic.


\end{Description}
If the discount parameter is set to 0, the distribution will become a symmetric Dirichlet, the LDA default.
\end{Description}

\subsection{Controlling sampling of hyperparameters}
\begin{Description}[\OptArg{-t}{transfile}]\setlength{\itemsep}{0cm}

\item[\OptArg{-F}{var}]
Fix the variable \texttt{var} where
it takes the value of a hyper-parameter to the \Opt{-S} option.
\item[\OptArg{-g}{var,int}]
Set the extra integer parameter for sampling \texttt{var} to \texttt{int}.
\item[\OptArg{-G}{var,cycles,start}]
Sample the variable \texttt{var} where
it takes the value of a hyper-parameter to the \Opt{-S} option. 
The sampling will be  after \texttt{start} cycles and then repeat every \texttt{cycles} cycles.

\end{Description}

\subsection{General control}
\begin{Description}[\OptArg{-t}{transfile}]\setlength{\itemsep}{0cm}
\item[\OptArg{-b}{epochs}] 
Specify the maximum number of previous epochs that table indicators can use for computation.
\item[\OptArg{-c}{cycles}] 
Do a checkpoint every this many \texttt{cycles}.
This saves the output statistics and the parameter file
adequate to do a restart with \Opt{-r} option.
\item[\OptArg{-C}{cycles}] 
Stop after this many \texttt{cycles}.
Default is 100. 
\item[\OptArg{-d}{dots}] 
For really big batches of data, print a 
``.'' every \texttt{dots} documents within a single cycle.
\item[\Opt{-e}]
Reroute logging to the \File{stderr}.
\item[\OptArg{-f}{format}] 
Read input data from data formatted according to
the type \texttt{format}.  Data is expected to come from
an input file with name \File{DataStem.Suff} where
\File{Suff} is an appropriate suffix.
These are given with Input Files below.
Allowed formats are:
\texttt{ldac}, \texttt{witdit}, \texttt{docword}, 
\texttt{bag}
and \texttt{lst}.
\item[\OptArg{-K}{topics}] 
Set $T$ the maximum number of topics.
Default is 10.
\item[\OptArg{-N}{maxN,maxM}] 
Set maximum for the Stirling number tables
to count \texttt{maxM} and maximum table counts for $a_{\mu}$ and $a_{\phi}$ to \texttt{maxM}.
Default is 1000,10000. Higher numbers are needed when running with large number of documents.
\item[\OptArg{-q}{threads}] If compiled with threading, enables
this many threads.  Default is 1.
\item[\Opt{-r}]
Restart from a previous run, reloading data and parameters.
\item[\OptArg{-R}]
Restart from the output file from \Prog{hca}
and the beta side is held constant and not sampled.
Can significantly speed up testing or querying sometimes.
\item[\OptArg{-s}{seed}]
Initialise the random number seed.
\item[\OptArg{-v}] Up verbosity by one increment.
Starts at zero and currently understands 0-3.
\item[\Opt{-V}]  load the dictionary from the
\File{DataStem.tokens} file for use in reporting.  It has one token per line.
\item[\OptArg{-W}{max}] Set to the maximum number of unique words to \texttt{max}
\end{Description}

\subsection{Testing and reports}
\begin{Description}[\OptArg{-t}{transfile}]\setlength{\itemsep}{0cm}
\item[\OptArg{-h}{Hold,arg}] 
Do document completion testing on the test set.
There are three styles of document completion implemented.
When \texttt{Hold} is the string ``doc'', then every \texttt{arg}-th
word is held out in estimating the latent variables (like theta)
for the document and used instead for testing of perplexity.
That is, words at document positions \texttt{arg-1}, \texttt{2*arg-1}, 
\emph{etc.}
When \texttt{Hold} is the string ``dict'', then every \texttt{arg}-th
word in the dictionary is held out in estimating
and used for testing.  So if a word has dictionary index 
\texttt{arg-1}, \texttt{2*arg-1}, \emph{etc.}, it is held out.
When \texttt{Hold} is the string ``fract'',
then the \texttt{fract}
proportion at the tail of the document is held out.  
The initial proportion is used in estimating.
\item[\OptArg{-l}{Diag,cycles,start}] 
Do a run-time estimation of the diagnostic \texttt{Diag}
starting after the \texttt{start} cycle and then taking the
estimate every \texttt{cycles} cycle.
Diagnostics are:
\begin{Description}[testprob]\setlength{\itemsep}{0cm}
\item[sp] 
Estimate topic sparsity in the theta matrix for the
words given in \File{DataStem.smap}.
Results placed in \File{RepStem.smap}.
The report gives ``topic/weight'' for topics including the word.
\item[prog] 
How often to do the standard diagnostic reports
(default is every 5-th cycle).
\item[phi] 
Estimate the word probability vector for each topic.
Stored in the \File{RepStem.phi} file.
\item[testprob] 
Estimate the topic probability vector for each test document.  
Stored in the \File{RepStem.testprob} file.
\item[theta] 
Estimate the topic probability vector for each training document.
Stored in the \File{RepStem.theta} file.
\end{Description}
\item[\OptArg{-L}{Diag,cycles,start}] 
Do a diagnostic estimate \texttt{Diag} after
all Gibbs sampling is complete.
Sampling of the estimate starts after the \texttt{start} cycle 
and goes for a total of \texttt{cycles} cycles
(including the starting ones).
Diagnostics are:
\begin{Description}[class]\setlength{\itemsep}{0cm}
\item[class] 
Estimate class probabilities with ``true'' classes
given in \File{DataStem.class} and then
produce confusion matrix for the test data.
Output to files
\File{DataStem.cnfs} and \File{DataStem.pcnfs}.
\item[like] 
Estimate likelihood/perplexity on the test set
using the standard (biased) document likelihood.
% or document completion if the \Opt{-h}
% option is used.
% Can also be instigated during run-time with the
% \Opt{-P} option.
\end{Description}
\item[\OptArg{-o}{score}]  Scoring rule to pick top words for printing.
Methods are `count', `idf', and `cost'.
\item[\OptArg{-p}] Report topic coherency in the log file, and 
save results in the parameter file.  This requires 
a \File{DataStem.pmi} or \File{DataStem.pmi.gz} file exist
in the right format.  This can be created with the 
\Prog{mkmat.pl} and
\Prog{cooc2pmi.pl} scripts in the scripts directory of the release.
The format is a simple sparse matrix form with lines
of the form ``N M PMI'' for word indices
(offset by 0) N and M and PMI value.
\item[\OptArg{-t}{size}]  Specify size of training set.  It takes the
first \texttt{size} entries in the data set. Default is all the
set minus the test data.
\item[\OptArg{-T}{size}]  Specify size of test set.  It takes the last
\texttt{size} entries as the test set.
Default is zero.
\item[\Opt{-V}]  load the dictionary from the
\File{DataStem.tokens} file for use in reporting.  It has one token per line.
\end{Description}

\section{Input Files}
%%%%%%%%%%%%%%%

The following files provide details about the dataset.
The filenames are constructed by adding a suffix to the data stem.
The data (document+word) format itself can be one of four different
formats and is specified with the \Opt{-f} option.
\begin{Description}\setlength{\itemsep}{0cm}
\item[\File{DataStem.class}] Class index for each document, one per line.  
Optional file used with some reports instigated by
\Opt{-X} or \OptArg{-L}{class} options.

\item[\File{DataStem.cnf}] This format appears in some UCI data sets
\item[\File{DataStem.docs}] A list of documents including their titles
\item[\File{DataStem.epoch}] epoch information file. The first number indicates the total number of epochs in the dataset, followed by the number of entries in each epoch in subsequent lines.
\item[\File{DataStem.ldac}] Standard LdaC format.  Word indices to the dictionary are offset by 1.
\item[\File{DataStem.srcpar}] summary information on the total number of documents and features of the dataset.
\item[\File{DataStem.stops}] List of stop words
\item[\File{DataStem.titles}] A list of titles for all entries
\item[\File{DataStem.tokens}] tokens/words in the dictionary, one per line.
\item[\File{DataStem.tpc}] Meta data of the dataset structure
\item[\File{DataStem.txtbag}] default bag or list format for \Cmd{linkBags}{1} command of \texttt{DCABags}.  Word indices offset by 0.
\item[\File{DataStem.words}] tokens/words in the dictionary, one per line.

\end{Description}

\section{Output Files}
%%%%%%%%%%%%%%%

The following files are output when the system checkpoints 
or at the end of the run.
These are built by adding a suffix to the report stem,
\File{RepStem}.
The first set of files are:
\begin{Description}\setlength{\itemsep}{0cm}
\item[\File{RepStem.cnfs}+\File{RepStem.pcnfs}]  
Best prediction and probability vector confusion matrices
built on the test data with the 
\OptArg{-L}{class} command.
\item[\File{RepStem.log}] Log file created if \Opt{-e} option not used.
\item[\File{RepStem.par}] Parameter and dimensions file in simple ``var = value'' format.  These are detailed in the next section.
\item[\File{RepStem.phi}] The Phi matrix written as a binary file:
first $W$ (dictionary size), $T$ (topics), 
$C$ (sample size) are written as 32 bit integers and
then the full Phi matrix as native floats with $W$ as the minor index.
Only generated with appropriate use of the
\OptArg{-l}{phi} option.
\item[\File{RepStem.smap}] Optional sparsity report on the 
word indices listed in \File{DataStem.smap}.
The report is instigated by the
\OptArg{-l}{sp} option.

\item[\File{RepStem.top}] A simple text report giving the top word indices
  for each topic.  If a hierarchical model in use, then the
``-1'' topic is for the base distribution of words.
Word indices are offset from 0.
\item[\File{RepStem.theta}] Estimated topic probabilities 
for each training document
written in a simple sparse form.  The class index
(``-1'' or ``+1'' for binary classes, otherwise just the index)
is also added if it exists.
Topic indices are offset by 0.
Only generated with appropriate use of the
\OptArg{-l}{theta} option.
\item[\File{RepStem.tpk}] Topic-word information per epoch.
\item[\File{RepStem.testprob}] 
Like the \texttt{-ltheta} option but for the test documents.
Only generated with appropriate use of the
\OptArg{-l}{testprob} option.
\end{Description}

The second set of files gives the actual runtime statistics.
Output matrices are in a simple readable sparse vector format
the same as the \File{DataStem.docword} format.
\begin{Description}\setlength{\itemsep}{0cm}
\item[\File{RepStem.ndt}] Document by topic counts.
\item[\File{RepStem.nwt}] Word by topic counts.
\item[\File{RepStem.tdt}] Document by topic table counts if
the Alpha side of the model is non-parametric.
\item[\File{RepStem.twt}] Word by topic table counts if
the Beta side of the model is non-parametric.
\item[\File{RepStem.zt}] With no burstiness, gives topic
index (offset by 0), one per line.  
With burstiness, gives one ``z,r'' per line where ``z'' is the
topic index (offset by 0) and ``r'' is the burst table indicator, 
which is 1 if the word
contributes to standard LDA statistics, and
0 if burstiness modelling says the word is a burst
so does not contribute to LDA statistics.
\end{Description}
These files along with \File{RepStem.par} are input
on a restart using \Opt{-r}.

\section{The Parameter File}

The parameter file has the following \emph{dimensions}:
\begin{Description}[T]
\item{N} -- number of words in the full collection,
          summed over all documents.
\item{NT} -- number of words in the training set,
          summed over all training documents.
\item{W} -- number of words in the dictionary.
\item{D} -- number of documents in total.
\item{TRAIN} -- number of documents to train on, is always the
the first ones in the file.
\item{TEST} -- number of documents to test on, is always the
the last ones in the file.
\item{T} -- maximum number of topics.
\item{ITER} -- number of major cycles made last.


\end{Description}

In addition, the float parameters allowed to be specified with the
\Opt{-F} and \Opt{-G} options are also given. This includes all the hyper-parameters for distribution $\mu$, $\phi$ and $\theta$ (\texttt{am}, \texttt{bm0}, \texttt{bm1}, \texttt{b0m}, \texttt{at}, \texttt{bt}, \texttt{ab} , \texttt{bb}, \texttt{ap0}, \texttt{ap1}, \texttt{b0p}, \texttt{bp1} ).


\section{Examples}
%%%%%%%%%%%%%%%%%%

These examples work as is on late model Linux, Macs and Windows.
However, you need to replace the executable,
\texttt{tca}, by the system dependent one,
from the install directory where the \File{data/} directory is.
For instance, on Windows that might be \texttt{tca/tca.exe}.

Run the topic model from \citet{dtmpypwl} with fixed hyper-parameters on the full dataset and no testing,
sending logging to \Prog{stderr}.
\begin{verbatim}
tca -v -e -V -K20 -C100 data/ch c1
\end{verbatim}
The parameters as they are sampled will be
printed on a line beginning wth ``Par:''.
At the end, the top 20 words will be printed and the
final training perplexity printed.  This is based on
the posterior probability, not word probability
estimates.
Restart and build a topic probability vector for each document,
as well as sparsity mappings for the words in 
\File{data/ch.smap} file.
\begin{verbatim}
tca -v -r -e -lsp,2,1 -ltheta,2,1 -C20 data/ch c1
\end{verbatim} 

Now view the sparsity report at \File{c1.smap} and
the topic probabilities at \File{c1.theta},
and the values saved in the parameter file \File{c1.par}. 
Rerun the model fitting on the full dataset and testing
on the final 100 documents.  Logging now to \File{c1.log}.
Checkpoint every 20 cycles
(note, we usually only do this for cycles taking over 10 minutes each).

\begin{verbatim}
tca -v -V -K20 -C100 -c20 -T100 data/ch c1
\end{verbatim}

View the end of the log file to get the test perplexity,
which is printed after ``log\_2(test perpML)''
Now rerun but use document completion (every 4th word), not the default
likelihood calculation.
\begin{verbatim}
tca -v -V -r -hdoc,4   data/ch c1
\end{verbatim}

View the end of the log file to get the test perplexity,
which is printed after ``log\_2(test perpHold)''.
Note it is also recorded in the parameter file.
Restart, no sampling this time using \OptArg{-C}{0}, just record the 
PMI and the classification details on test data.
\begin{verbatim}
tca -v -V -r -C0 -Llike,0,0 -Lclass,10,1 -p  data/ch c1
\end{verbatim}

Note the \OptArg{-L}{like,0,0} option is used to prevent it 
doing test likelihood calculations, which are potentially slow
on larger data sets.
The PMI data has a value printed for each topic as well as a 
final average.  It bases its calcuations on the matrix
\File{data/ch/pmi.gz} created explicitly for this test set.
For other datasets, you will need to down load prepared
PMI matrices from the project homepage.

To relax the assumptions on the fixed hyperparameters, one can run with 
the optimisation options. For instance, if we allow the word-topic distribution (\texttt{phi}) to 
evolve at different speed for different topics, we can sample the concentration parameter \texttt{bp0} and \texttt{bp1} as the following example,
\begin{verbatim}
tca -v -v -K20 -C200 -Gbp0,2,2 -Gbp1,2,2 data/ch c1
\end{verbatim}
The hyperparameter sampling slows it down quite a bit but seems to
make a significant difference.  Unused topics sometimes
get a very low concentration.

\section{Errors}

There is some error reporting on failure.

If the software quits during a run on larger data with an
error message like:
\begin{verbatim}
    S_V(N,M,A) tagged 'XXX' hit bounds (BN,BM)
\end{verbatim}
for integers \texttt{N,M} and label \texttt{XXX} then you
need to increase the bounds.
If the tag \texttt{XXX} is ``SX, docXtopic PYP"
then increase the bound \texttt{BM} using the option
\OptArg{-N}{BM} with an increased \texttt{BM}.
If the tag \texttt{XXX} is ``SX, docXtopic PYP"
then increase both bounds \texttt{BN,BM} using the option
\OptArg{-N}{BM,BN}
(note the order of the bounds).

For other errors, please report to the
maintainer.

\section{See Also}
%%%%%%%%%%%%%%%%%%


The command \Cmd{linkBags}{1} is available from \URL{http://mloss.org} in 
the software \Prog{DCABags}.
The extended library \Prog{libstb}, parts of which are included, is available
individually from \URL{http://mloss.org}.

Other supporting software and data sets are available at the project home
page at\\
\URL{https://github.com/wbuntine/topic-models}

\section{Version}
%%%%%%%%%%%%%%%%%

This programme is version \Version\ of \Date.
This incorporates parts of the library \Prog{libstb} version 1.8
also of \Date.

\section{License and Copyright}
%%%%%%%%%%%%%%%%%%%%%%%%%%%%%%%

\begin{description}
\item[Copyright] \copyright\ 2011-2014, Prof.~Wray Buntine, 
  NICTA, Canberra, Australia (to 2013), and Monash
University (from 2014)\\
     \Email{wray.buntine@monash.edu}
Some parts also by Dr.\ Jinjing Li (2013) and 
Mr.\ Swapnil Mishra (2013-2014).

\item[License]  This Source Code Form is subject to the terms of the Mozilla 
 Public License, v. 2.0. If a copy of the MPL was not
 distributed with this file, You can obtain one at
      \URL{http://mozilla.org/MPL/2.0/}.
\end{description}

\section{Author}
%%%%%%%%%%%%%%%%

\noindent
Prof.~Wray Buntine                     \\
Email: \Email{Wray.Buntine@monash.edu}  

Some parts also done by Dr.\ Jinjing Li and 
Mr.\ Swapnil Mishra.

\LatexManEnd

\begin{thebibliography}{breitestes Label}
\bibitem[Buntine and Li(2014)]{dtmpypwl}Buntine, W. and Li, J. (2014). Dynamic Topic Models using Hierarchical
Two-parameter Poisson-Dirichlet Process
\end{thebibliography}
\end{document}

